\documentclass[12pt]{article}
\usepackage[utf8]{inputenc}

\title{AEP4380 HW8 \\ Least Squares Curve Fitting }
\author{Dan Girshovich}
\date{April 12, 2013}

\usepackage{graphicx}
\usepackage{subfigure}
\usepackage{amsmath}
\usepackage{float}
\usepackage{courier}
\usepackage[margin=1.0in]{geometry}
\usepackage{listings}
\usepackage{xcolor}

\lstset {
    language=C++,
    backgroundcolor=\color{black!5}, % set backgroundcolor
    basicstyle=\footnotesize,% basic font setting
}

\begin{document}

\maketitle
\section{Overview}
This document details the results of the included C++ program, which uses a reduced Chi-squared metric to fit a linear combination of functions to provided data. Specifically, the atmospheric CO2 data from the last 50 years is fit to a combination of polynomial and sinusoidal terms. Seasonal variations in the data can be removed using this method simply by discounting the sinusoidal terms.
\section{Fitting Method}
\subsection{Overview}
A linear fitting method seeks to find a function
\begin{align*}
f(t, \vec{a}) = \sum_{k = 1}^{m} a_k f_k(t)
\end{align*}
such that the value
\begin{align*}
\chi_r^2 = \frac{1}{N - m} \sum_{i = 1}^{N} \left( \frac{y_i - f(t_i)}{\sigma_i} \right)^2
\end{align*}
is minimized. Here, the $m$ functions $f_k$ are fitting functions, the $m$ values $a_k$ are fitting parameters, and the reduced Chi-squared value $\chi_r^2$ is a function of the fit $f(t, \vec{a})$, the data $\vec{y}$, and the errors in the data $\vec{\sigma_i}$.
\subsection{Matrix Formulation}
Solving for the optimal fitting parameters $\vec{a}$ in the above equations is equivalent to solving for $\vec{a}$ in
\begin{align*}
\mathbf{F}\vec{a} = \vec{b}
\end{align*}
where
\begin{align*}
F_{lk} = \sum_{i=1}^{N} \frac{f_l(t_i) f_k(t_i)}{\sigma_i^2}
\text{  and  }
b_l = \sum_{i = 1}^{N} \frac{y_i f_l(t_i)}{\sigma_i^2}.
\end{align*}
$\mathbf{F}$ and $\vec{b}$ are calculated in \texttt{get\_F} and \texttt{get\_b}, respectively. \texttt{gaussj} from Numerical Recipes Chapter 2 \cite{numericalrecipes} is then used to perform Gauss - Jordan elimination (with pivoting) to solve for $\vec{a}$ given $\mathbf{F}$ and $\vec{b}$. \texttt{gaussj} also produces $\mathbf{F^{-1}}$, which contains the \emph{parameter} errors in its diagonals.
\section{Fitting to Data}
\subsection{Carbon Dioxide Measurements}
\begin{figure}
  \centering
  \includegraphics[scale=0.5]{orig.png}
  \caption{Atmospheric C02 measurements}
  \label{f1}
\end{figure}
The CO2 data on \texttt{http://cdiac.ornl.gov/ftp/trends/co2/maunaloa.co2} is shown in Figure \ref{f1}. Notice that there are short-term seasonal variations as well as a long-term upward trend. This data is read into the program through \texttt{load\_arrays}.
\subsection{Fitting Functions}
A set of fitting function are linearly combined (with the fitting parameters) to create the fit. Looking at the data, it is evident that the long-term trend can be approximated with a (roughly linear) polynomial term and the short term trend can be approximated with sinusoidal terms. From this observation, the following fitting functions were chosen ($t$ has units of months):
\begin{align*}
1 \text{, } t \text{, } t^2 \text{, }
sin(kt) \text{, } cos(kt) \text{, }
sin(\frac{kt}{2}) \text{, and } cos(\frac{kt}{2})
\end{align*}
where $k = \frac{2\pi}{12}$, so the oscillating terms represent variations on both the $1$ year and $6$ month scale. \\ \\
These functions are created in \texttt{get\_fit\_funcs}. This implementation allows the selection of a subset of fitting functions through the flag \texttt{first\_harmonic}, which specifies whether the oscillating terms with the $6$ month period are included. The function returns a \texttt{std:vector} of fitting functions that is used to solve for fitting parameters and analyze the fit by the remainder of the code.
\subsection{Goodness of Fit Information}
The fitting parameters are found using the matrix formulation described. This is implemented in \texttt{write\_all\_fit\_data}, which uses \texttt{gaussj} to solve for $\vec{a}$. This function also extracts errors from $\mathbf{F_{-1}}$, calculates the final fit function, uses it to find $\chi_r^2$, and outputs the results.
\section{Results}
\subsection{Fits}
The data was fit using the polynomial and sinusoidal basis functions given in the last section. The results are shown in Figure \ref{f2}. The fit was also done without the seasonal variation terms included in the final fit function, and this is shown in Figure \ref{f3}. The fit with seasonal variations is shown as an overlay of the original data in Figure \ref{f4}. Finally, a residual plot between the full fit and the data is shown in Figure \ref{f5}.
\subsection{Goodness of Fit}
For the full fit, $\chi_r^2 = 0.000174229$, which indicates a very good fit. The residual plot has low correlation, also indicating a good fit. However the residuals seem to show a long term oscillating term could be added to the fit function to further reduce correlation. The final fit parameters and their errors are shown below:
\begin{center}
\begin{tabular}{ c | c | c }
$f$ & $a$ & $error$ \\ \hline
$1$ & 314.148 & 64.0772 \\
$t $ & 0.0668684 & 0.00391019 \\
$t^2$ & 8.40915e-05 & 1.02474e-08 \\
$sin(kt)$ & 2.77709 & 15.5171 \\
$cos(kt)$ & -0.365962 & 15.5776 \\
$sin(\frac{kt}{2})$ & 0.0197292 & 15.4916 \\
$cos(\frac{kt}{2})$ & 0.0301944 & 15.6276 \\
\end{tabular}
\end{center}
\begin{figure}
  \centering
  \includegraphics[scale=0.5]{fit7.png}
  \caption{Fit using all fitting functions.}
  \label{f2}
\end{figure}
\begin{figure}
  \centering
  \includegraphics[scale=0.5]{fit3.png}
  \caption{Fit with seasonal variations removed.}
  \label{f3}
\end{figure}
\begin{figure}
  \centering
  \includegraphics[scale=0.5]{fit_on_data.png}
  \caption{Fit as an overlay of the original data.}
  \label{f4}
\end{figure}
\begin{figure}
  \centering
  \includegraphics[scale=0.5]{resid.png}
  \caption{Residuals between full fit and original data.}
  \label{f5}
\end{figure}
\subsection{Without First Harmonic}
A fit without the 6 month periodic terms produces a fit with $\chi_r^2 = 0.000173787$, which indicates a slightly better fit than the full 7 parameter fit. This means that there is likely no oscillation in the data with a 6 month period. The fact that the fit parameters for these first harmonic terms is relatively small in the full fit also supports this claim.
\subsection{Source Code}
\begin{lstlisting}
/*
AEP 4380 HW #8
Dan Girshovich
4/12/13
Least Squares Curve Fitting
Compile with: g++ -std=c++0x -O2 -o gen_data hw8.cpp
Tested on Mac OSX 10.8.2 with a Intel Core 2 Duo
*/

#include <cstdio>
#include <cstdlib>
#include <cmath>
#include <algorithm>
#include <functional>
#include "nr3.h"

using namespace std;

typedef function<double(double)> math_func;
const int N = 607; // max number of data points
const double sigma_perc =  0.2;
vector<int> ts = vector<int>(N);
vector<double> ys = vector<double>(N);
vector<double> sigmas = vector<double>(N);

// slightly modified given code to read from the data file
void load_arrays() {
    const int NCMAX = 200; // max number of characters per line
    char cline[N];
    double co2;
    FILE *fp;
    fp = fopen("maunaloa.co2", "r");
    if (NULL == fp) {
        printf("Can't open file.");
        exit(0);
    }
    for (int i = 0; i < 15; i++) fgets(cline, NCMAX, fp); // read a whole line
    int t = 0, npts = 0;
    int year;
    do {
        fscanf(fp, "%d", &year);
        for (int j = 0; j < 12; j++) {
            fscanf(fp, "%lf", &co2);
            if (co2 > 0.0) {
                ts[npts] = t;
                ys[npts++] = co2;
            }
            t++;
        }
        if (npts >= N) break;
        fscanf(fp, "%lf", &co2); // skipping average value
    } while (year < 2008);
}

// returns an array of fit functions
vector<math_func> get_fit_funcs(bool first_harmonic) {
    const double pi = 4.0 * atan(1.0);
    double k = 2 * pi / 12;
    vector<math_func> fs {
        [ = ](double t){return 1;},
        [ = ](double t){return t;},
        [ = ](double t){return t * t;},
        [ = ](double t){return sin(k * t);},
        [ = ](double t){return cos(k * t);},
    };
    vector<math_func> half_year_terms {
        [ = ](double t) {return sin((k * t) / 2);},
        [ = ](double t) {return cos((k * t) / 2);},
    };
    if (first_harmonic) {
        fs.insert(fs.end(), half_year_terms.begin(), half_year_terms.end());
    }
    return fs;
}

// returns the F matrix given fit functions
MatDoub_IO get_F(vector<math_func> fs) {
    int m = fs.size();
    MatDoub_IO F(m, m);
    for (int l = 0; l < m; l++) {
        for (int k = 0; k < m; k++) {
            double sum = 0.0;
            for (int i = 0; i < N; i++) {
                double y = ys[i];
                sum += fs[l](ts[i]) * fs[k](ts[i]) / (sigmas[i] * sigmas[i]);
            }
            F[l][k] = sum;
        }
    }
    return F;
}

// returns the b vector given fit functions
MatDoub_IO get_b(vector<math_func> fs) {
    int m = fs.size();
    MatDoub_IO b(m, 1);
    for (int l = 0; l < m; l++) {
        double sum = 0.0;
        for (int i = 0; i < N; i++) {
            double y = ys[i];
            sum += y * fs[l](ts[i]) / (sigmas[i] * sigmas[i]);
        }
        b[l][0] = sum;
    }
    return b;
}

// From Numerical Recipes Ch 2
void gaussj(MatDoub_IO &a, MatDoub_IO &b) {
    // see text
}

// From Numerical Recipes Ch 2
void gaussj(MatDoub_IO &a) {
    MatDoub b(a.nrows(), 0);
    gaussj(a, b);
}

void write1(string name, int suffix, function<double(int)> g, int max) {
    stringstream fname;
    fname << name << suffix << ".dat";
    ofstream of;
    of.open(fname.str().c_str());
    for (int i = 0; i < max; i++) {
        of << g(i) << endl;
    }
    of.close();
}

void write2(string name, int suffix, function<double(int)> g, int max) {
    stringstream fname;
    fname << name << suffix << ".dat";
    ofstream of;
    of.open(fname.str().c_str());
    for (int i = 0; i < max; i++) {
        of << ts[i] << " " << g(i) << endl;
    }
    of.close();
}

void write_all_fit_data(vector<math_func> fs, bool seasonal_var) {
    MatDoub_IO F = get_F(fs);
    MatDoub_IO b = get_b(fs);
    gaussj(F, b);
    // now b contains the solution vector (a), and F is (old F)^-1
    MatDoub &F_inv = F;
    MatDoub &a = b;
    int m = seasonal_var ? fs.size() : 3;
    math_func f = [ = , &a](double t) {
        double sum = 0.0;
        for (int k = 0; k < m; k++) {
            sum += a[k][0] * fs[k](t);
        }
        return sum;
    };
    auto get_params = [&a](int i) {
        return a[i][0];
    };
    write1("params", m, get_params, m);
    auto get_fit = [f](int i) {
        return f(ts[i]);
    };
    write2("fit", m, get_fit, N);
    auto get_error = [F_inv](int i) {
        return F_inv[i][i];
    };
    write1("error", m, get_error, m);
    auto get_chi_sq = [f, m](int j) {
        double sum = 0.0;
        for (int i = 0; i < N; i++) {
            double tmp = (ys[i] - f(ts[i])) / sigmas[i];
            sum += tmp * tmp;
        }
        return sum / (N - m);
    };
    write1("chi_sq", m, get_chi_sq, 1);
}

int main() {
    load_arrays();
    transform(ys.begin(), ys.end(), sigmas.begin(), [](double y) {
        return y * sigma_perc;
    });
    auto get_orig_data = [](int i) {
        return ys[i];
    };
    write2("orig", 0, get_orig_data, N);
    // with first harmonic
    vector<math_func> fs = get_fit_funcs(true);
    write_all_fit_data(fs, true);
    // no first harmonic
    vector<math_func> fs2 = get_fit_funcs(false);
    write_all_fit_data(fs2, true);
    // with seasonal variations removed after fit calculation
    write_all_fit_data(fs, false);
}
\end{lstlisting}

\begin{thebibliography}{9}

\bibitem{numericalrecipes}
  W.H. Press, S.A Teukolsky, W. T. Vetterling, and B. P. Flannery,
  \emph{Numerical Recipies, The Art of Scientific Computing}.
  Camb. Univ. Press,
  3rd Edition,
  2007.

\end{thebibliography}

\end{document}