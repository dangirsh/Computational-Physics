\documentclass[12pt]{article}
\usepackage[utf8]{inputenc}

\usepackage{amsmath}
\usepackage{courier}

\begin{document}

\begin{center}
Final Project Outline \\
Three-Body Simulations \\
Dan Girshovich
\end{center}

For my final project, I'd like to use numerical methods to simulate and analyze the recently discovered solutions to the three-body problem. \\

Earlier this year, 13 new solutions to the three-body problem were discovered and presented along with a solution classification scheme \cite{suv}. As a first challenge, I'd like to apply an automatic step size Runge-Kutta method (like one in Numerical Recipes \cite{numericalrecipes}) to recreate each of the new solution orbits. The initial conditions are available online \cite{3bsite}, so the main challenge would be to maintain accuracy over long orbits and produce 3D plots. \\

The classification scheme involves a ``shape space sphere'' and is also described online \cite{3bsite}. Collision conditions and the orbits are represented as points and contours on this sphere, respectively. The symmetry properties of these contours with respect to collision points allows for a group theory based categorization of the solution orbits. A second goal for the project will be to generate these ``shape space sphere'' plots for the solutions and briefly describe the classification scheme. For completeness, I will also include the few previously known solutions.\\

The stability of the new solutions is still being explored. I plan on performing a shallow investigation using a Monte Carlo method. Specifically, repeated simulations involving random perturbations to the orbits should give some basic insight into the stability of the solutions. The analysis will include a discussion on how the automatic step size algorithm handles the different types of perturbations. On the same note, any signs of instability will be persuasively, but informally, shown to be a property of the solution itself, and not the Runge-Kutta method.\\

Finally, because these 13 solutions are relatively new, it is expected that slight variations of some of them will produce yet undiscovered solutions \cite{suv}. All 13 are solutions where the three bodies have equal mass and zero total angular momentum. I can implement a basic algorithm that greedily searches for alternate choices of mass and total angular momentum which produce solution orbits (those with finite periods). I realize that any simple method is extremely unlikely to discover a new solution, so I'm putting this challenge under the ``if time'' category.

\begin{thebibliography}{9}

\bibitem{suv}
    M. Šuvakov and V. Dmitrašinović, Three classes of Newtonian three-body periodic orbits, Phys. Rev. Lett. 110 (2013) 114301

\bibitem{3bsite}
    http://suki.ipb.ac.rs/3body/

\bibitem{numericalrecipes}
  W.H. Press, S.A Teukolsky, W. T. Vetterling, and B. P. Flannery,
  \emph{Numerical Recipies, The Art of Scientific Computing}.
  Camb. Univ. Press,
  3rd Edition,
  2007.

\end{thebibliography}

\end{document}
