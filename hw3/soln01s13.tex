%
%   AEP 438 HW 1 solutions
%
%   spring 2013
%
%  22-jan-2002 E. Kirkland
%  changed date 27-jan-2003 ejk
%  changed date 28-jan-2004 ejk
%  changed date and switch to dvipdfm 26-jan-2005 ejk
%  changed date 24-jan-2006 ejk
%  changed date 29-jan-2007 ejk
%  changed date 26-jan-2009 ejk
%  changed date 27-jan-2010 ejk
%  fix some grammar issues 2-feb-2010 ejk
%  last modified 22-jan-203 ejk
%
\documentclass[11pt,letterpaper]{article}

\setlength{\textwidth}{6.5in}
\setlength{\textheight}{9in}
\setlength{\topmargin}{-0.5in}
\setlength{\oddsidemargin}{0in}
\setlength{\evensidemargin}{0in}

\setlength{\parskip}{0.1in}
\setlength{\parindent}{0.5in}

%\pagestyle{empty}  % get rid of page numbers
\pagestyle{myheadings}
\markboth{{ AEP4380}, Solution 1, Spring 2013}
{{ AEP4380}, Solution 1, Spring 2013}

\usepackage[dvipdfm]{graphics}  % for complicated illustrations
%\usepackage[pdftex]{graphicx}  % for complicated illustrations with non-eps graphics
%\usepackage[dvips]{graphics}  % for complicated illustrations


\begin{document}

\begin{center}
\underline{\bf \large Numerical Differentiation }
\end{center}

{\em
    You are not required to turn in a written solution to this homework problem.  
However, I will still hand out a solution set to give you an idea of the format of the 
homework solutions that will come later.  Note that the solution set that is handed out 
may occasionally contain a longer discussion of some subjects than is formally required.
}

Numerical differentiation techniques will be tested on the following function:
\begin{eqnarray}
f(x) & = & \sin(x) \exp ( -0.04x^2 )
\end{eqnarray}
The first derivative of $f(x)$, $f'(x)$ may be calculated three different way:
\begin{eqnarray}
f_{FD}'(x) & = & \frac{ f(x+h) - f(x) }{h} + \mathcal{O}(h) \mbox{~~~forward difference} \\
f_{BD}'(x) & = & \frac{ f(x) - f(x-h) }{h} + \mathcal{O}(h)  \mbox{~~~backward difference} \\
f_{CD}'(x) & = & \frac{ f(x+h) - f(x-h) }{2h} + \mathcal{O}(h^2) 
                \mbox{~~~central difference} 
\end{eqnarray}
where $h$ is small compared to $x$. The forward and backward difference methods may have 
a significantly larger error than the central difference method.  Alternately the forward and 
backward difference may be viewed as a central difference method offset by 
+h/2 and -h/2 respectively.  $f'(x)$ from all three methods was evaluated at 200 points 
between -7.0 and +7.0  using h=0.5 and h=0.05.  These three forms of the first derivative 
were calculated using the program listed at the end, and the results are shown
below along with the function $f(x)$ in Figure \ref{fbigh} and \ref{fsmallh} 
(FD= forward difference,  BD= 
backward difference, CD= central difference). The program was run twice, once with 
h=0.5 and once with h=0.05.  (The program also calculates the second derivative that 
follows.)

As can be seen from the Figure \ref{fbigh},  h=0.5 is too large and the three methods yield very 
different results.  The forward difference (FD) and the backward difference (BD)
appear to be shifted consistent with their interpretation as a shifted central difference.  
The second plot (Figure \ref{fsmallh}) with h=0.05 shows that this value of h is small
enough to yield an accurate (and consistent) calculation of the first derivative.

\begin{figure}[htb!]
\begin{center}
\leavevmode
\resizebox{4.7in}{!}{\includegraphics{fig1a.eps}}
\end{center}
\caption{Finite first difference derivatives with h=0.5.
\label{fbigh}}
\end{figure}

         
\begin{figure}[htb!]
\begin{center}
\leavevmode
\resizebox{4.7in}{!}{\includegraphics{fig1b.eps}}
\end{center}
\caption{Finite first difference derivatives with h=0.05.
\label{fsmallh}}
\end{figure}

The second derivative may also be calculated numerically as:
\begin{eqnarray}
f''(x) & = & \frac{ f(x+h) -2f(x) + f(x-h) }{h^2} + \mathcal{O}(h^2)
\end{eqnarray}
The results are shown in the in Figure \ref{ffpp} below. In this range of $x$ the
$\exp()$ factor of
$f(x)$  does not change much so that the second derivative should be approximately:
\begin{eqnarray}
f''(x) & \sim & \exp ( -0.04x^2 ) \times \frac{\partial^2}{\partial x^2}
    \left[ \sin(x) \right] = - f(x)
\end{eqnarray}
             
\begin{figure}[htb!]
\begin{center}
\leavevmode
\resizebox{4.7in}{!}{\includegraphics{fig1c.eps}}
\end{center}
\caption{Second derivative with finite differences  (h=0.05).
\label{ffpp}}
\end{figure}

This means that the second derivative is similar to the negative of $f(x)$, which is
indeed apparent.

\newpage 

\begin{center}
\underline{\bf \large Source Code Listing}
\end{center}

\begin{quote} \footnotesize
\begin{verbatim}
/*  hw01s13.cpp  AEP 438 Homework # 1

    Calculate numerical derivatives of function feval()
    using the forward difference, backward difference
    and central difference  

	Run on a Pentium with gcc 4.6

	E. Kirkland 22-Jan-2013	
*/

#include <cstdio>
#include <cstdlib>
#include <cmath>

int main()
{
    int i, n=200;
    double h=0.5, xmin=-7, xmax=+7.0, x, dx;
//  double h=0.05, xmin=-7, xmax=+7.0, x, dx;
    double f1, f2, f3, fpbd, fpfd, fpcd, fpp;
    double feval( double );
    FILE* fp;

    /* open new file for output and write a title */

    fp = fopen( "hw01.dat", "w+");
    if( NULL == fp ) {
        printf( "cannot open file\n" );
        return( 0 );
    }
//  fprintf( fp, "    X    F(X)     FPFD    FPBD    FPCD     FPP \n");

    dx = ( xmax - xmin ) / (n-1);
    for( i=0; i<n; i++){
       x = xmin + i * dx;
       f1 = feval( x - h );
       f2 = feval( x );
       f3 = feval( x + h );
       fpbd = (f2 - f1)/h;
       fpfd = (f3 - f2) /h;
       fpcd = ( f3 - f1 ) / ( 2.0 * h );
       fpp = ( f3 - 2.0*f2 + f1 ) / ( h*h );
       fprintf(fp, "%16.8f %16.8f %16.8f %16.8f %16.8f %16.8f \n",
            x, f2, fpfd, fpbd, fpcd, fpp);
       printf("%16.8f %16.8f %16.8f %16.8f %16.8f %16.8f \n",
            x, f2, fpfd, fpbd, fpcd, fpp);
    }   
    fclose( fp );

    return( EXIT_SUCCESS );
}

/*-------------------- feval ------------------------------
/*
   function to differentiate
*/
double feval( double x )
{   return(  sin(x) * exp( -0.04*x*x ) );  }
\end{verbatim}
\end{quote}

\end{document}
